% Options for packages loaded elsewhere
\PassOptionsToPackage{unicode}{hyperref}
\PassOptionsToPackage{hyphens}{url}
%
\documentclass[
]{article}
\usepackage{lmodern}
\usepackage{amssymb,amsmath}
\usepackage{ifxetex,ifluatex}
\ifnum 0\ifxetex 1\fi\ifluatex 1\fi=0 % if pdftex
  \usepackage[T1]{fontenc}
  \usepackage[utf8]{inputenc}
  \usepackage{textcomp} % provide euro and other symbols
\else % if luatex or xetex
  \usepackage{unicode-math}
  \defaultfontfeatures{Scale=MatchLowercase}
  \defaultfontfeatures[\rmfamily]{Ligatures=TeX,Scale=1}
\fi
% Use upquote if available, for straight quotes in verbatim environments
\IfFileExists{upquote.sty}{\usepackage{upquote}}{}
\IfFileExists{microtype.sty}{% use microtype if available
  \usepackage[]{microtype}
  \UseMicrotypeSet[protrusion]{basicmath} % disable protrusion for tt fonts
}{}
\makeatletter
\@ifundefined{KOMAClassName}{% if non-KOMA class
  \IfFileExists{parskip.sty}{%
    \usepackage{parskip}
  }{% else
    \setlength{\parindent}{0pt}
    \setlength{\parskip}{6pt plus 2pt minus 1pt}}
}{% if KOMA class
  \KOMAoptions{parskip=half}}
\makeatother
\usepackage{xcolor}
\IfFileExists{xurl.sty}{\usepackage{xurl}}{} % add URL line breaks if available
\IfFileExists{bookmark.sty}{\usepackage{bookmark}}{\usepackage{hyperref}}
\hypersetup{
  hidelinks,
  pdfcreator={LaTeX via pandoc}}
\urlstyle{same} % disable monospaced font for URLs
\usepackage[margin=1in]{geometry}
\usepackage{longtable,booktabs}
% Correct order of tables after \paragraph or \subparagraph
\usepackage{etoolbox}
\makeatletter
\patchcmd\longtable{\par}{\if@noskipsec\mbox{}\fi\par}{}{}
\makeatother
% Allow footnotes in longtable head/foot
\IfFileExists{footnotehyper.sty}{\usepackage{footnotehyper}}{\usepackage{footnote}}
\makesavenoteenv{longtable}
\usepackage{graphicx,grffile}
\makeatletter
\def\maxwidth{\ifdim\Gin@nat@width>\linewidth\linewidth\else\Gin@nat@width\fi}
\def\maxheight{\ifdim\Gin@nat@height>\textheight\textheight\else\Gin@nat@height\fi}
\makeatother
% Scale images if necessary, so that they will not overflow the page
% margins by default, and it is still possible to overwrite the defaults
% using explicit options in \includegraphics[width, height, ...]{}
\setkeys{Gin}{width=\maxwidth,height=\maxheight,keepaspectratio}
% Set default figure placement to htbp
\makeatletter
\def\fps@figure{htbp}
\makeatother
\setlength{\emergencystretch}{3em} % prevent overfull lines
\providecommand{\tightlist}{%
  \setlength{\itemsep}{0pt}\setlength{\parskip}{0pt}}
\setcounter{secnumdepth}{-\maxdimen} % remove section numbering
\usepackage{setspace}\onehalfspacing
\usepackage{lineno}
\linenumbers
\usepackage{float}
\usepackage{booktabs}
\usepackage{pdflscape}
\newcommand{\blandscape}{\begin{landscape}}
\newcommand{\elandscape}{\end{landscape}}
\usepackage{caption}
\captionsetup[table]{font=small}
\captionsetup[figure]{font=small}
\captionsetup[table]{labelformat=empty}
\captionsetup[figure]{labelformat=empty}
\usepackage{dcolumn}
\usepackage[]{natbib}
\bibliographystyle{apalike}

\author{}
\date{\vspace{-2.5em}}

\begin{document}

\raggedright

\textbf{Title:} Tree height and leaf drought tolerance traits shape
growth responses across droughts in a temperate broadleaf forest

\textbf{Authors:} Ian R. McGregor\textsuperscript{1,2}, Ryan
Helcoski\textsuperscript{1}, Norbert Kunert\textsuperscript{1,3}, Alan
J. Tepley\textsuperscript{1,4}, Erika B.
Gonzalez-Akre\textsuperscript{1}, Valentine Herrmann\textsuperscript{1},
Joseph Zailaa\textsuperscript{1,5}, Atticus E.L.
Stovall\textsuperscript{1,6,7}, Norman A. Bourg\textsuperscript{1},
William J. McShea\textsuperscript{1}, Neil Pederson\textsuperscript{8},
Lawren Sack\textsuperscript{9,10}, Kristina J.
Anderson-Teixeira\textsuperscript{1,3}*

\textbf{Author Affiliations:}

\begin{enumerate}
\def\labelenumi{\arabic{enumi}.}
\tightlist
\item
  Conservation Ecology Center; Smithsonian Conservation Biology
  Institute; National Zoological Park, Front Royal, VA 22630, USA
\item
  Center for Geospatial Analytics; North Carolina State University;
  Raleigh, NC 27607, USA
\item
  Center for Tropical Forest Science-Forest Global Earth Observatory;
  Smithsonian Tropical Research Institute; Panama, Republic of Panama
\item
  Canadian Forest Service, Northern Forestry Centre, Edmonton, Alberta,
  Canada
\item
  Biological Sciences Department; California State University; Los
  Angeles, CA 90032, USA
\item
  Department of Environmental Sciences, University of Virginia,
  Charlottesville, VA 22903, USA
\item
  NASA Goddard Space Flight Center; Greenbelt, MD 20771, USA
\item
  Harvard Forest, Petersham, MA 01366, USA
\item
  Department of Ecology and Evolutionary Biology; University of
  California, Los Angeles; Los Angeles, CA 90095, USA
\item
  Institute of the Environment and Sustainability; University of
  California, Los Angeles; Los Angeles, CA 90095, USA
\end{enumerate}

*corresponding author:
\href{mailto:teixeirak@si.edu}{\nolinkurl{teixeirak@si.edu}}; +1 540 635
6546

\begin{longtable}[]{@{}llll@{}}
\toprule
\begin{minipage}[b]{0.35\columnwidth}\raggedright
Text\strut
\end{minipage} & \begin{minipage}[b]{0.16\columnwidth}\raggedright
word count\strut
\end{minipage} & \begin{minipage}[b]{0.22\columnwidth}\raggedright
other\strut
\end{minipage} & \begin{minipage}[b]{0.15\columnwidth}\raggedright
n\strut
\end{minipage}\tabularnewline
\midrule
\endhead
\begin{minipage}[t]{0.35\columnwidth}\raggedright
Total word count (excluding summary, references and legends)\strut
\end{minipage} & \begin{minipage}[t]{0.16\columnwidth}\raggedright
5,365\strut
\end{minipage} & \begin{minipage}[t]{0.22\columnwidth}\raggedright
No.~of figures\strut
\end{minipage} & \begin{minipage}[t]{0.15\columnwidth}\raggedright
4 (all colour)\strut
\end{minipage}\tabularnewline
\begin{minipage}[t]{0.35\columnwidth}\raggedright
Summary\strut
\end{minipage} & \begin{minipage}[t]{0.16\columnwidth}\raggedright
198\strut
\end{minipage} & \begin{minipage}[t]{0.22\columnwidth}\raggedright
No.~of Tables\strut
\end{minipage} & \begin{minipage}[t]{0.15\columnwidth}\raggedright
3\strut
\end{minipage}\tabularnewline
\begin{minipage}[t]{0.35\columnwidth}\raggedright
Introduction\strut
\end{minipage} & \begin{minipage}[t]{0.16\columnwidth}\raggedright
1,034\strut
\end{minipage} & \begin{minipage}[t]{0.22\columnwidth}\raggedright
No of Supporting Information files\strut
\end{minipage} & \begin{minipage}[t]{0.15\columnwidth}\raggedright
\#\strut
\end{minipage}\tabularnewline
\begin{minipage}[t]{0.35\columnwidth}\raggedright
Materials and Methods\strut
\end{minipage} & \begin{minipage}[t]{0.16\columnwidth}\raggedright
1,945\strut
\end{minipage} & \begin{minipage}[t]{0.22\columnwidth}\raggedright
\strut
\end{minipage} & \begin{minipage}[t]{0.15\columnwidth}\raggedright
\strut
\end{minipage}\tabularnewline
\begin{minipage}[t]{0.35\columnwidth}\raggedright
Results\strut
\end{minipage} & \begin{minipage}[t]{0.16\columnwidth}\raggedright
697\strut
\end{minipage} & \begin{minipage}[t]{0.22\columnwidth}\raggedright
\strut
\end{minipage} & \begin{minipage}[t]{0.15\columnwidth}\raggedright
\strut
\end{minipage}\tabularnewline
\begin{minipage}[t]{0.35\columnwidth}\raggedright
Discussion\strut
\end{minipage} & \begin{minipage}[t]{0.16\columnwidth}\raggedright
1467\strut
\end{minipage} & \begin{minipage}[t]{0.22\columnwidth}\raggedright
\strut
\end{minipage} & \begin{minipage}[t]{0.15\columnwidth}\raggedright
\strut
\end{minipage}\tabularnewline
\begin{minipage}[t]{0.35\columnwidth}\raggedright
Acknowledgements\strut
\end{minipage} & \begin{minipage}[t]{0.16\columnwidth}\raggedright
125\strut
\end{minipage} & \begin{minipage}[t]{0.22\columnwidth}\raggedright
\strut
\end{minipage} & \begin{minipage}[t]{0.15\columnwidth}\raggedright
\strut
\end{minipage}\tabularnewline
\bottomrule
\end{longtable}

\newpage

\hypertarget{summary}{%
\subsubsection{Summary}\label{summary}}

\begin{itemize}
\item
  As climate change is driving increased drought frequency and severity
  in many forested regions around the world, mechanistic understanding
  of the factors conferring drought resistance in trees is increasingly
  important. The dendrochronological record provides a window through
  which we can understand how tree size and species' traits shape tree
  growth responses during droughts.
\item
  We analyzed tree-ring records for twelve species that comprise 97\% of
  the woody productivity of the 25.6-ha ForestGEO plot in a broadleaf
  deciduous forest of northern Virginia (USA) to test hypotheses on how
  tree height, microenvironment characteristics, and species' traits
  shaped drought responses across the three strongest regional droughts
  over a 60-year period (1950 - 2009).
\item
  Individual-level drought resistance decreased with tree height, which
  was strongly correlated with exposure to higher evaporative demand and
  solar radiation. The potentially greater rooting volume of larger
  trees did not confer an advantage in sites with low topographic
  wetness index. Resistance was greater among species whose leaves
  experienced less shrinkage upon desiccation and lost turgor (wilted)
  at more negative water potentials.
\item
  We conclude that tree height and leaf drought tolerance traits
  influence growth responses during drought, as recorded in the
  tree-ring record spanning historical droughts. Thus, these factors can
  be useful for predicting future drought responses under climate
  change.
\end{itemize}

\emph{Key words}: annual growth; crown exposure; drought; Forest Global
Earth Observatory (ForestGEO); leaf drought tolerance traits; temperate
broadleaf deciduous forest; tree height; tree-ring

\newpage

\hypertarget{introduction}{%
\subsubsection{Introduction}\label{introduction}}

Forests play a critical global role in climate regulation
\citep{bonan_forests_2008}, yet there remains enormous uncertainty as to
how the forest-dominated terrestrial carbon sink will respond to climate
change \citep{friedlingstein_climatecarbon_2006}. An important aspect of
this uncertainty lies with physiological responses of trees to drought
\citep{kennedy_implementing_2019}. In many forested regions around the
world, the risk of severe drought is increasing
\citep{trenberth_global_2014, dai_climate_2018}, often despite
increasing precipitation
\citep{intergovernmental_panel_on_climate_change_climate_2015, cook_unprecedented_2015}.
Droughts, intensified by climate change, have been affecting forests
worldwide and are expected to continue as one of the most important
drivers of forest change in the future
\citep{allen_global_2010, allen_underestimation_2015}. Understanding
forest responses to drought requires elucidation of how tree size,
microenvironment, and species' traits jointly influence individual-level
drought resistance, and the extent to which their influence is
consistent across droughts. Because the resistance and resilience of
growth to drought is linked to trees' probability of surviving drought
\citep{desoto_low_2020, liu_hydraulic_2019}, understanding growth
responses can also help elucidate which trees are most vulnerable to
drought-induced mortality. However, it has proven difficult to resolve
the many factors affecting tree growth during drought with available
forest census data, which only rarely captures extreme drought, and with
tree-ring records, which capture multiple droughts but usually only
sample a subset of a forest community, typically focusing on a single
species or the largest individuals.

Many studies have shown that within and across species, large trees tend
to be more affected by drought. Greater growth reductions for larger
trees were first shown on a global scale by \citet{bennett_larger_2015},
and subsequent studies have reinforced this finding
\citep[\emph{e.g.},][]{hacket-pain_consistent_2016, gillerot_tree_2020}.
It has yet to be resolved which of several potential underlying
mechanisms most strongly shape these trends in drought response. First,
tree height itself may be a primary driver. Taller trees face the
biophysical challenge of lifting water greater distances against the
effects of gravity and friction
\citep{mcdowell_relationships_2011, mcdowell_darcys_2015, ryan_hydraulic_2006, couvreur_water_2018}.
Vertical gradients in stem and leaf traits--including smaller and
thicker leaves (higher leaf mass per area, LMA), greater resistance to
hydraulic dysfunction (\emph{i.e.}, more negative water potential at
50\% loss of hydraulic conductivity, more negative P50), and lower
hydraulic conductivity at greater heights
\citep{couvreur_water_2018, koike_leaf_2001, mcdowell_relationships_2011}--enable
trees to become tall \citep{couvreur_water_2018}. Greater stem
capacitance (\emph{i.e.}, water storage capacity) of larger trees may
also confer resistance to transient droughts
{[}\citet{phillips_reliance_2003}*; \citet{scholz_hydraulic_2011}{]}.
Indeed, tall trees require xylem of greater hydraulic efficiency, such
that xylem conduit diameters are wider in the basal portions of taller
trees, both within and across species
\citep{olson_plant_2018, liu_hydraulic_2019}, and throughout the
conductive systems of angiosperms
\citep{zach_vessel_2010, olson_universal_2014, olson_plant_2018}. Wider
xylem conduits plausibly make large trees more vulnerable to embolism
during drought \citep{olson_plant_2018}, and traits conducive to
efficient water transport may also lead to poor ability to recover from
or re-route water around embolisms \citep{roskilly_conflicting_2019}.

Larger trees may also have lower drought resistance because of
microenvironmental and ecological factors. Their crowns tend to occupy
more exposed canopy positions, which are associated with higher
evaporative demand \citep{kunert_revised_2017}. Subcanopy trees tend to
fare better specifically due to the benefits of a buffered environment
\citep{pretzsch_drought_2018}. Counteracting the liabilities associated
with tall height, large trees tend to have larger root systems
\citep{enquist_global_2002}, potentially mitigating some of the
biophysical challenges they face by allowing greater access to water.
Larger root systems--if they grant access to deeper water sources--would
be particularly advantageous in drier microenvironments (e.g., hilltops,
as compared to valleys and streambeds) during drought. Finally, tree
size-related responses to drought can be modified by species' traits and
their distribution across size classes
\citep{meakem_role_2018, liu_hydraulic_2019}. Understanding the
mechanisms driving the greater relative growth reductions of larger
trees during drought requires sorting out the interactive effects of
height and associated exposure, root water access, and species' traits.

Debates have also arisen regarding the traits influencing tree growth
responses to drought. Studies within temperate broadleaf forests have
observed ring-porous species showing higher drought tolerance than
diffuse-porous species
\citep{friedrichs_species-specific_2009, elliott_forest_2015, kannenberg_linking_2019},
but this distinction would not hold in the global context
\citep{wheeler_variations_2007, olson_xylem_2020} and does not resolve
differences among the many species within each category.
Commonly-measured traits including wood density and leaf mass per area
(\(LMA\)) have been linked to drought responses within some temperate
deciduous forests
\citep{abrams_adaptations_1990, guerfel_impacts_2009, hoffmann_hydraulic_2011, martinbenito_convergence_2015}
and across forests worldwide \citep{greenwood_tree_2017}. However, in
other cases these traits could not explain drought tolerance
\citep[e.g., in a tropical rainforest;][]{marechaux_leaf_2019}, or the
direction of response was not always consistent. For instance, higher
wood density has been associated with greater drought resistance at a
global scale \citep{greenwood_tree_2017}, but correlated negatively with
tree performance during drought in a broadleaf deciduous forest in the
southeastern United States \citep{hoffmann_hydraulic_2011}. Thus, the
perceived influence of these traits on drought resistance may actually
reflect indirect correlations with other traits that more directly drive
drought responses \citep{hoffmann_hydraulic_2011}.

In contrast, hydraulic traits have direct physiological linkages to tree
growth and mortality responses to drought. For instance, water
potentials at which percent the loss of conductivity surpasses a certain
threshold (e.g., P50 and P88, representing 50 and 88\% loss of
conductivity, respectively) and hydraulic safety margin (\emph{i.e.},
difference between typical minimum water potentials and P50 or P88)
correlate with drought performance across global forests
\citep{anderegg_meta-analysis_2016}. However, these are time-consuming
to measure and therefore infeasible for predicting or modeling drought
responses in highly diverse forests (\emph{e.g.}, in the tropics). More
easily-measurable leaf drought tolerance traits that have direct linkage
to plant hydraulic function can explain variation in plant distribution
and function \citep{medeiros_extensive_2019}. These include leaf area
shrinkage upon desiccation \citep[\(PLA_{dry}\);][]{scoffoni_leaf_2014}
and the leaf water potential at turgor loss point (\(\pi_{tlp}\)),
\emph{i.e.,} the water potential at which leaf wilting occurs
\citep{bartlett_correlations_2016, zhu_leaf_2018}. Both traits correlate
with hydraulic vulnerability and drought tolerance as part of unified
plant hydraulic systems
\citep{scoffoni_leaf_2014, bartlett_correlations_2016, zhu_leaf_2018, farrell_does_2017}.
The abilities of both \(PLA_{dry}\) and \(\pi_{tlp}\) to explain tree
drought resistance remains untested.

Here, we examine how tree height, microenvironment characteristics, and
species' traits collectively shape drought resistance, defined as the
ratio of annual growth in a drought year to that which would be expected
in the absence of drought based on previous years' growth. We test a
series of hypotheses and associated specific predictions (Table 1) based
on the combination of tree-ring records from the three strongest
droughts over a 60-year period (1950 - 2009), species trait
measurements, and census and microenvironmental data from a large forest
dynamics plot in Virginia, USA. First, we focus on how tree size, alone
and in its interaction with microenvironmental gradients, influences
drought resistance. We examine the contemporary relationship between
tree height and microenvironment, including growing season
meteorological conditions and crown exposure. We then test whether,
consistent with most forests globally, larger-diameter, taller trees
tend to have lower drought resistance in this forest, which is in a
region (eastern North America) represented by only two studies in the
global review of Bennett et al.~(2015). We also test for an influence of
potential access to available soil water, which should be greater for
larger trees in dry but not in perpetually wet microsites. Finally, we
focus on the role of species' traits, testing the hypothesis that
species' traits-\/--particularly leaf leaf drought tolerance
traits-\/--predict drought resistance. We test predictions that drought
resistance is higher in ring-porous than semi-ring and diffuse-porous
species and that it is correlated with wood density--either positively
\citep{greenwood_tree_2017} or negatively
\citep{hoffmann_hydraulic_2011} and positively correlated with \(LMA\).
We further test predictions that species with low \(PLA_{dry}\) have
higher drought resistance, and that species whose leaves lose turgor
lower water potentials (more negative \(\pi_{tlp}\)) have higher
resistance.

\hypertarget{materials-and-methods}{%
\subsubsection{Materials and Methods}\label{materials-and-methods}}

\emph{Study site and microclimate}

Research was conducted at the 25.6-ha ForestGEO (Forest Global Earth
Observatory) study plot at the Smithsonian Conservation Biology
Institute (SCBI) in Virginia, USA (38°53'36.6``N, 78°08'43.4''W; Fig.
\textbf{S1})
\citep{bourg_initial_2013, andersonteixeira_ctfs-forestgeo:_2015}. SCBI
is located in the central Appalachian Mountains near the northern
boundary of Shenandoah National Park. Elevations range from 273 to 338 m
above sea level with a topographic relief of 65m
\citep{bourg_initial_2013}. Climate is humid temperate, with mean annual
temperature of 12.7\textsuperscript{\(\circ\)}C and precipitation of
1005 mm yr\textsuperscript{-1} during our study period \citep[1960-2009;
source: CRU TS v.4.01;][]{harris_updated_2014}. Dominant tree taxa
within this secondary forest include \emph{Liriodendron tulipifera},
oaks (\emph{Quercus} spp.), and hickories (\emph{Carya} spp.; Table 2).

\emph{Identifying drought years}

We identified the three largest droughts within the time period
1950-2009, defining drought \citep{slette_how_2019} as events with
anomalously dry peak growing season climatic conditions. Specifically,
we used the metric of Palmer Drought Severity Index (PDSI) during
May-August (MJJA; Table S1), which were identified by
\citet{helcoski_growing_2019} as the months of the current year to which
annual tree growth was most sensitive at this site. PDSI divisional data
for Northern Virginia were obtained from NOAA
(\url{https://www7.ncdc.noaa.gov/CDO/CDODivisionalSelect.jsp}) in
December 2017. Based on this, we identified the three strongest droughts
during the study period (Figs. \textbf{1, S1}; Table S1).

The droughts differed in intensity and antecedent moisture conditions
(Fig. \textbf{S1}, Table S1). The 1966 drought was preceded by two years
of moderate drought during the growing season and severe to extreme
drought starting the previous fall. In August 1966, \(PDSI\) reached its
lowest monthly value (-4.82) of the three droughts. The 1977 drought was
the least intense throughout the growing season, and it was preceded by
2.5 years of near-normal conditions, making it the mildest of the three
droughts. The 1999 drought was preceded by wetter than average
conditions until the previous June, but \(PDSI\) plummeted below -3.0 in
October 1998 and remained below this threshold through August 1999.

\emph{Data collection and preparation}

Within or just outside the ForestGEO plot, we collected data on a suite
of variables including tree heights, microenvironment characteristics,
and species traits (Table 3). The SCBI ForestGEO plot was censused in
2008, 2013, and 2018 following standard ForestGEO protocols, whereby all
free-standing woody stems \(\ge\) 1cm diameter at breast height (DBH)
were mapped, tagged, measured at DBH, and identified to species
\citep{condit_tropical_1998}. From these census data, we used
measurements of DBH from 2008 to calculate historical DBH and data for
all stems \(\ge\) 10cm to analyze functional trait composition relative
to tree height (all analyses described below). Census data are available
through the ForestGEO data portal (www.forestgeo.si.edu).

We analyzed tree-ring data (xylem growth increment) from 571 trees
representing the twelve dominant species (Table 2; Fig. \textbf{S2}).
Selected species were those with the greatest contributions to woody
aboveground net primary productivity (\(ANPP_{stem}\)) and together
comprised 97\% of study plot \(ANPP_{stem}\) between 2008 and 2013
\citep{helcoski_growing_2019}. Cores (one per tree) were collected
within the ForestGEO plot at breast height (1.3m) in 2010-2011 or
2016-2017. In 2010-2011, cores were collected from randomly selected
live trees of each species that had at least 30 individuals \(\ge\) 10
cm DBH \citep{bourg_initial_2013}. Annual tree mortality censuses were
initiated in 2014 \citep{gonzalezakre_patterns_2016}, and in 2016-2017,
cores were collected from all trees found to have died since the
previous year's census. We note that drought was probably not a cause of
mortality for these trees, as monthly May-Aug \(PDSI\) did not drop
below -1.75 in these years or the three years prior (2013-2017), and
that trees cored dead displayed similar climate sensitivity to trees
cored live \citep{helcoski_growing_2019}. Cores were sanded, measured,
and crossdated using standard procedures, as detailed in
\citep{helcoski_growing_2019}. The resulting chronologies (Fig.
\textbf{1a}) were published in Zenodo (DOI: 10.5281/zenodo.2649302) in
association with \citet{helcoski_growing_2019}.

For each cored tree, we combined tree-ring records and allometric
equations of bark thickness to reconstruct DBH for the years 1950-2009.
Prior \(DBH\) was estimated using the following equation:

\[DBH_Y  = DBH_{2008} - 2*\left[r_{bark,2008} - r_{bark,Y} + \sum_{year=Y}^{2008} r_{ring, Y} \right]\]

Here, \(Y\) denotes the year of interest, \(r_{ring}\) denotes ring
width derived from cores, and \(r_{bark}\) denotes bark thickness. Bark
thickness was estimated from species-specific allometries based on the
bark thickness data from the site
\citep{andersonteixeira_size-related_2015}. Specifically, we used linear
regression on log-transformed data to relate \(r_{bark}\) to diameter
inside bark from 2008 data (Table S2), which were then used to determine
\(r_{bark}\) in the \(DBH\) reconstruction.

Tree heights (\(H\)) were measured by several researchers for a variety
of purposes between 2012 and 2019 (n=1,518 trees). Methods included
direct measurements using a collapsible measurement rod on small trees
\citep{neon_national_2018} or a tape measure on recently fallen trees
(this study); geometric calculations using clinometer and tape measure
\citep{stovall_assessing_2018} or digital rangefinders
\citep{andersonteixeira_size-related_2015, neon_national_2018}; and
ground-based LiDAR \citep{stovall_terrestrial_2018}. Rangefinders used
either the tangent method (Impulse 200LR, TruPulse 360R) or the sine
method (Nikon ForestryPro) for calculating heights. Both methods are
associated with some error \citep{larjavaara_measuring_2013}, but in
this instance there was no clear advantage of one or the other.
Measurements from the National Ecological Observatory Network (NEON)
were collected near the ForestGEO plot following standard NEON protocol,
whereby vegetation of short stature was measured with a collapsible
measurement rod, and taller trees with a rangefinder
\citep{neon_national_2018}. Species-specific height allometries were
developed (Table S3) using log-log regression (\(ln[H] \sim ln[DBH]\)).
For species with insufficient height data to create reliable
species-specific allometries (n=2, JUNI and FRAM), heights were
calculated from an equation developed by combining the height
measurements across all species. We then used these allometries to
estimate \(H\) for each drought year, \(Y\), based on reconstructed
\(DBH_Y\). The distribution of \(H\) across drought years is shown in
Fig. \textbf{S3}.

To characterize how environmental conditions vary with height, data were
obtained from the NEON tower located \textless1km from the study area
via the neonUtilities package \citep{R-neonUtilities}. We used wind
speed, relative humidity, and air temperature data, all measured over a
vertical profile spanning heights from 7.2 m to above the top of the
tree canopy (31.0 or 51.8m, depending on censor), for the years
2016-2018 \citep{neon_national_2018}. After filtering for missing and
outlier values, we determined the daily minima and maxima, which we then
aggregated at the monthly scale.

Crown position--a categorical variable classifying trees based on
exposure to sunlight--was recorded for all cored trees that remained
standing during the growing season of 2018 following the protocol of
\citet{jennings_assessing_1999}. Trees were classified as follows:
\emph{dominant} trees were defined as those with crowns above the
general level of the canopy, \emph{co-dominant} trees as those with
crowns within the the canopy; \emph{intermediate} trees as those with
crowns below the canopy level, but illuminated from above; and
\emph{suppressed} as those below the canopy and receiving minimal direct
illumination from above.

Topographic wetness index (TWI), used here as a metric of long-term mean
moisture availability, was calculated using the dynatopmodel package in
R (Fig. \textbf{S2}) \citep{R-dynatopmodel}. Originally developed by
\citet{beven_physically_1979}, TWI was part of a hydrological run-off
model and has since been used for a number of purposes in hydrology and
ecology \citep{sorensen_calculation_2006}. TWI calculation depends on an
input of a digital elevation model (DEM; \textasciitilde3.7 m resolution
from the elevatr package \citep{R-elevatr}), and from this yields a
quantitative assessment defined by how ``wet'' an area is, based on
areas where run-off is more likely. From our observations in the plot,
TWI performed better at categorizing wet areas than the Euclidean
distance from the stream.

Species' trait data were collected in August 2018 (Tables 2-3; Fig.
\textbf{S4}). We sampled small, sun-exposed branches up to eight meters
above the ground from three individuals of each species in and around
the ForestGEO plot. Sampled branches were re-cut under water at least
two nodes above the original cut and re-hydrated overnight in covered
buckets under opaque plastic bags before measurements were taken.
Rehydrated leaves taken towards the apical end of the branch (n=3 per
individual: small, medium, and large) were scanned, weighed, dried at
60\(^\circ\) C for \(\ge\) 48 hours, and then re-scanned and weighed.
Leaf area was calculated from scanned images using the LeafArea R
package \citep{R-LeafArea}. \(LMA\) was calculated as the ratio of leaf
dry mass to fresh area. \(PLA_{dry}\) was calculated as the percent loss
of area between fresh and dry leaves. Wood density was calculated for
\textasciitilde1cm diameter stem samples (bark and pith removed) as the
ratio of dry weight to fresh volume, which was estimated using
Archimedes' displacement. We used the rapid determination method of
\citet{bartlett_rapid_2012} to estimate osmotic potential at turgor loss
point (\(\pi_{tlp}\)). Briefly, two 4 mm diameter leaf discs were cut
from each leaf, tightly wrapped in foil, submerged in liquid nitrogen,
perforated 10-15 times with a dissection needle, and then measured using
a vapour pressure osmometer (VAPRO 5520, Wescor, Logan, UT, USA).
Osmotic potential (\(\pi_{osm}\)) given by the osmometer was used to
estimate (\(\pi_{tlp}\)) using the equation
\(\pi_{tlp}=0.832 \pi_{osm} ^{-0.631}\) \citep{bartlett_rapid_2012}.

\emph{Statistical Analysis}

For each drought year, we calculated a metric drought resistance
(\(Rt\)) as the ratio of basal area increment (\(BAI\); \emph{i.e.},
change in cross-sectional area) during the drought year to the mean
\(BAI\) over the five years preceding the drought
\citep{lloret_components_2011}. Thus, \(Rt\) values \textless1 and
\textgreater1 indicate growth reductions and increases, respectively.
Because the \(Rt\) metric could be biased by directional pre-drought
growth trends, we also tried an intervention time series analysis
(ARIMA, \citep{R-forecast}) that predicted mean drought-year growth
based on trends over the preceding ten years and used this value in
place of the five-year mean in calculations of resistance
(\(Rt_{ARIMA}\)= observed \(BAI\)/ predicted \(BAI\)). The two metrics
were strongly correlated (Fig. \textbf{S5}). Visual review of the
individual tree-ring sequences with the largest discrepancies between
these metrics revealed that \(Rt\) was less prone to unreasonable
estimates than \(Rt_{ARIMA}\), so we selected \(Rt\) as our focal
metric, presenting parallel results for \(Rt_{ARIMA}\) in the
Supplementary Info. In this study we focus exclusively on drought
resistance metrics (\(Rt\) or \(Rt_{ARIMA}\)), and not on the resilience
metrics described in \citet{lloret_components_2011}, because (1) we
would expect resilience to be controlled by a different set of
mechanisms, and (2) the findings of \citep{desoto_low_2020} suggest that
\(Rt\) is a more important drought response metric for angiosperms in
that low resistance to moderate droughts was a better predictor of
mortality during subsequent severe droughts than the resilience metrics.

Analyses focused on testing the predictions presented in Table 1 with
\(Rt\) as the response variable, and then repeated using \(Rt_{ARIMA}\)
as the response variable. Models were run for all drought years combined
and for each drought year individually. The general statistical model
for hypothesis testing was a mixed effects model, implemented in the
lme4 package in R \citep{R-lme4}. In the multi-year model, we included a
random effect of tree nested within species and a fixed effect of
drought year to represent the combined effects of differences in drought
characteristics. Individual year models included a random effect of
species. All models included fixed effects of independent variables of
interest (Tables 1,3) as specified below. All variables across all best
models had variance inflation factors \textless1.2 (1 +/- 0.019). We
used AICc to assess model selection, and conditional/marginal R-squared
to assess model fit as implemented in the AICcmodavg package in R
\citep{R-AICcmodavg}. AICc refers to a corrected version of AICc, and is
best suited for small data sizes \citep[see][]{brewer_relative_2016}.

To avoid over-fitting models with five species traits (Table 3) across
only 12 species, we did not include all traits as fixed effects in a
single linear mixed model, but rather conducted individual tests of each
species trait to determine the relative importance and appropriateness
for inclusion in the main model. These tests followed the model
structure specified above, then added \(ln[H]\) and \(ln[TWI]\) to
create a base model against which we tested traits. Trait variables were
considered appropriate for inclusion in the main model if they had a
consistent direction of response across all droughts and if their
addition to the base model improved fit (at \(\Delta\)AICc \(\ge\) 1.0)
in at least one drought year (Table S4). We note that we did not use the
\(\Delta\)AICc \(\ge\) 1.0 criterion as a test of significance, but
rather of whether the variable had enough influence to be considered as
a \emph{candidate} variable in full models.

We then determined the top full models for predicting \(Rt\) (or
\(Rt_{ARIMA}\)). To do so, we compared models with all possible
combinations of candidate variables, including \(ln[H]\)*\(ln[TWI]\) and
species traits as specified above. We identified the full set of models
within \(\Delta\)AICc=2 of the best model (that with lowest AICc). When
a variable appeared in all of these models and the sign of the
coefficient was consistent across models, we viewed this as support for
the acceptance/rejection of the associated prediction (Table 1). If the
variable appeared in some but not all of these models, and its sign was
consistent across models, we considered this partial support/rejection.
In presentation of the results below, we note instances where the
\(Rt_{ARIMA}\) model disagreed with the \(Rt\) model, but otherwise do
not discuss the \(Rt_{ARIMA}\) model.

All analysis beyond basic data collection was performed using R version
3.6.2 \citep{R-base}. Other R-packages used in analyses are listed in
the Supplementary Information (\emph{Appendix S1}). All data, code, and
results are available through the SCBI-ForestGEO organization on GitHub
(\url{https://github.com/SCBI-ForestGEO}: SCBI-ForestGEO-Data and
McGregor\_climate-sensitivity-variation repositories), with static
versions corresponding to data and analyses presented here archived in
Zenodo (DOIs: 10.5281/zenodo.3604993 and \emph{{[}TBD{]}}, respectively.

\hypertarget{results}{%
\subsubsection{Results}\label{results}}

\emph{Tree height and microenvironment}

In the years for which we have vertical profiles in climate data
(2016-2018), taller trees--or those in dominant crown positions-- were
generally exposed to higher evaporative demand during the peak growing
season months (May-August; Fig. \textbf{2}). Specifically, maximum daily
wind speeds were significantly higher above the top of the canopy
(40-50m) than within and below (10-30m) (Fig. \textbf{2a}). Relative
humidity was also somewhat lower during June-August, ranging from
\textasciitilde50-80\% above the canopy and \textasciitilde60-90\% in
the understory (Fig. \textbf{2b}). Air temperature did not vary
consistently across the vertical profile (Fig. \textbf{2c}).

Crown position varied as expected with height (dominant \textgreater{}
co-dominant \textgreater{} intermediate \textgreater{} suppressed), but
with substantial variation (Fig. \textbf{2d}). There were significant
differences in height across all crown position classes (Fig.
\textbf{2d}). A comparison test between height and crown position data
from the most recent ForestGEO census (2018) revealed a correlation of
0.73.

\emph{Community-level drought responses}

At the community level, cored trees showed substantial growth reductions
in all three droughts, with a mean \(Rt\) of 0.86 in 1966 and 1999, and
0.84 in 1977 (Fig. \textbf{2b}). Across the entire study period
(1950-2009), the focal drought years were the three years with the
largest fraction of trees exhibiting \(Rt \le 0.7\). Specifically, in
each drought, roughly 30\% of the cored trees had growth reductions of
\(\ge\) 30\% (\(Rt \le 0.7\)): 29\% in 1966, 32\% in 1977, and 27\% in
1999. However, some individuals exhibited increased growth, \emph{i.e.},
\(Rt > 1.0\): 26\% of trees in 1966, 22\% in 1977, and 26\% in 1999.

In the context of the multivariate model, \(Rt\) did not vary across
drought years. That is, drought year as a variable did not appear in any
of the top models -- \emph{i.e.}, models that were statistically
indistinguishable (\(\Delta\)AICc\textless2) from the best model.

\emph{Tree height, microenvironment, and drought resistance}

Taller trees (based on \(H\) in the drought year) showed stronger growth
reductions during drought (Table 1; Figs. \textbf{4, S6}). Specifically,
\(ln[H]\) appeared, with a negative coefficient, in the best model
((\(\Delta\)AICc=0) and all top models when evaluating the three drought
years together (Tables S6-S7). The same held true for 1966 individually.
For the 1977 drought, \(ln[H]\) did not appear in the best model, but
was included, with a negative coefficient, among the top
models--\emph{i.e.}, models that were statistically indistinguishable
(\(\Delta\)AICc\textless2) from the best model (Tables 1, S6-S7). For
the 1999 drought, \(ln[H]\) had no significant effect.

\(Rt\) had a significantly negative response to \(ln[TWI]\) across all
drought years combined (Figs. \textbf{4, S6}, Table S6-S7). The effect
was also significant for 1977 and 1999 individually (Fig. \textbf{4},
Table S6). When \(Rt_{ARIMA}\) was used as the response variable, the
effect was significant in 1977, and included in some of the top models
in 1966 and 1999 (Table S7). This negates the idea that trees in moist
microsites would be less affected by drought. Nevertheless, we tested
for a \(ln[H] *ln[TWI]\) interaction, a negative sign of which could
indicate that smaller trees (presumably with smaller rooting volume) are
more susceptible to drought in drier microenvironments with a deeper
water table. This hypothesis was rejected, as the \(ln[H] *ln[TWI]\)
interaction was never significant, and had a positive sign in any top
models in which it appeared (Tables 1, S6-S7). This term did appear with
a positive coefficient in the best \(Rt_{ARIMA}\) model for all years
combined (Table S7), indicating that the negative effect of height on
\(Rt\) was significantly stronger in wetter microhabitats.

\emph{Species' traits and drought resistance}

Species, as a factor in ANOVA, had significant influence
(p\textless0.05) on all traits (wood density, \(LMA\), \(PLA_{dry}\),
and \(\pi_{tlp}\)), with more significant pairwise differences for wood
density and \(PLA_{dry}\) than for \(LMA\) and \(\pi_{tlp}\) (Table 2,
Fig. \textbf{S4}). Drought resistance also varied across species,
overall and in each drought year (Fig. \textbf{3}). Significant
differences in \(Rt\) across species were most pronounced in 1966 with a
total of seven distinct groupings, while 1977 had four and 1999 had two.
Averaged across all droughts, \(Rt\) was lowest in \emph{Liriodendron
tulipifera} (mean \(Rt\) = \textbf{0.66}) and highest in \emph{Fagus
grandifolia} (mean \(Rt\) = \textbf{0.99}).

Wood density, \(LMA\), and xylem porosity were all poor predictors of
\(Rt\) (Tables 1,S4-S5). Wood density and \(LMA\) were never
significantly associated with \(Rt\) in the single-variable tests and
were therefore excluded from the full models. Xylem porosity was also
excluded from the full models, as it had no significant influence for
all droughts combined and had contrasting effects in the individual
droughts: whereas ring-porous species had higher \(Rt\) than diffuse-
and semi-ring- porous species in the 1966 and 1999 droughts, they had
lower \(Rt\) in 1977 (Table S4). It is noteworthy that the two
diffuse-porous species in our study, \emph{Liriodendron tulipifera} and
\emph{Fagus grandifolia}, were at opposite ends of the \(Rt\) spectrum
(Fig. \textbf{3}), further refuting the idea that xylem porosity is a
useful predictor of \(Rt\) in the context of this study.

In contrast, \(PLA_{dry}\), and \(\pi_{tlp}\) were both negatively
correlated to drought resistance (Figs. \textbf{4, S6}; Tables 1,S4-S7).
Both had consistent signs across all droughts, and their inclusion at
least marginally improved the model (\(\Delta\)AICc \textgreater{} 1.0)
for at least one of the three droughts (Table S4), qualifying them as
candidate variables for the full model. \(PLA_{dry}\) had a significant
influence, with negative coefficient, in full models for the three
droughts combined and for the 1966 drought individually (Fig.
\textbf{4}; Tables S6-S7). For 1977 and 1999, it was included with a
negative coefficient in some of the top models (Tables S6-S7).
\(\pi_{tlp}\) was included with a negative coefficient in the best model
for both all droughts combined and for the 1977 drought individually
(Fig. \textbf{4}; Table S6). It was also included in some of the top
models for 1999 (Tables S6-S7).

\hypertarget{discussion}{%
\subsubsection{Discussion}\label{discussion}}

Tree height, microenvironment, and leaf drought tolerance traits shaped
tree growth responses across three droughts at our study site (Table 1,
Fig. \textbf{4}). The greater susceptibility of larger trees to drought,
similar to forests worldwide \citep{bennett_larger_2015}, was driven
primarily by their height \citep{stovall_tree_2019}. Taller height was
likely a liability in itself, and was also associated with greater
exposure to conditions that would promote water loss and heat damage
during drought (Fig. \textbf{2}). There was no evidence that greater
availability of, or access to, soil water availability increased drought
resistance; in contrast, trees in wetter topographic positions had lower
\(Rt\) \citep{zuleta_drought-induced_2017, stovall_tree_2019}, and the
larger potential rooting volume of large trees provided no advantage in
the drier microenvironments. The negative effect of height on \(Rt\)
held after accounting for species' traits, which is consistent with
recent work finding height had a stronger influence on mortality risk
than forest type during drought \citep{stovall_reply_2020}. Drought
resistance was not consistently linked to species' \(LMA\), wood
density, or xylem type (ring- vs.~diffuse porous), but was negatively
correlated with leaf drought tolerance traits (\(PLA_{dry}\),
\(\pi_{tlp}\)). This is the first study to our knowledge linking
\(PLA_{dry}\) and \(\pi_{tlp}\) to growth reduction during drought. The
directions of these responses were consistent across droughts (Table
S6), supporting the premise that they were driven by fundamental
physiological mechanisms. However, the strengths of each predictor
varied across droughts (Fig. \textbf{4}; Tables S6-S7), indicating that
drought characteristics interact with tree size, microenvironment, and
traits to shape which individuals are most affected. These findings
advance our knowledge of the factors that make trees vulnerable to
growth declines during drought and, by extension, likely make them more
vulnerable to mortality \citep{sapes_plant_2019}.

The droughts considered here were of a magnitude that has occurred with
an average frequency of approximately once every 10-15 years \citep[Fig.
\textbf{1a},][]{helcoski_growing_2019} and had substantial but not
devastating impacts on tree growth (Figs. \textbf{1b}). These droughts
were classified as severe (\(PDSI\) \textless{} -3.0; 1977) or extreme
(\(PDSI\) \textless{} -4.0; 1966, 1999) at our site and have been linked
to tree mortality in the eastern United States
\citep{druckenbrod_redefining_2019}. However, extreme, multiannual
droughts such as the so-called ``megadroughts'' of this type that have
triggered massive tree die-off in other regions
\citep[e.g.,][]{allen_global_2010, stovall_tree_2019} have not occurred
in the Eastern United States within the past several decades
\citep{clark_impacts_2016}. Of the droughts considered here, the 1966
drought, which was preceded by two years of dry conditions (Fig.
\textbf{S1}), severely stressed a larger portion of trees (Fig.
\textbf{1b}). The tendency for large trees to have lowest resistance was
most pronounced in this drought, consistent with other findings that
this physiological response increases with drought severity
\citep{bennett_larger_2015, stovall_tree_2019}. Across all three
droughts, the majority of trees experienced reduced growth, but a
substantial portion had increased growth (Fig. \textbf{1b}), potentially
due to decreased leaf area of competitors during the drought
(\textbf{REF--if we can find one}), and consistent with prior
observations that smaller trees can exhibit increased growth rates
during drought \citep{bennett_larger_2015}. It is likely because of the
moderate impact of these droughts, along with other factors influencing
tree growth (e.g., stand dynamics), that our best models characterize
only a modest amount of variation in \(Rt\): 11-12\% for all droughts
combined, and 18-25\% for each individual drought (Fig. \textbf{S6};
Table S6).

Consistent with studies in other forests worldwide
\citep{bennett_larger_2015}, taller trees in this forest exhibited lower
drought resistance. Mechanistically, this is consistent with, and
reinforces, previous findings that biophysical constraints make it
impossible for trees to efficiently transport water to great heights and
simultaneously maintain strong resistance and resilience to
drought-induced embolism
\citep{olson_plant_2018, couvreur_water_2018, roskilly_conflicting_2019}.
Taller trees also face dramatically different microenvironments (Fig.
\textbf{2}). They are exposed to higher wind speeds and lower humidity
(Fig. \textbf{2a-b}), resulting in higher evaporative demand. Unlike
other temperate forests where modestly cooler understory conditions have
been documented \citep{zellweger_seasonal_2019}, particularly under
drier conditions \citep{davis_microclimatic_2019}, we observed no
significant variation in air temperatures across the vertical profile
(Fig. \textbf{2c}). More critically for tree physiology, leaf
temperatures can become significantly elevated over air temperature
under conditions of high solar radiation and low stomatal conductance
\citep{campbell_introduction_1998, rey-sanchez_spatial_2016}. Under
drought, when air temperatures tend to be warmer, direct solar radiation
tends to be higher (because of less cloud cover), and less water is
available for evaporative cooling of the leaves, trees with sun-exposed
crowns may not be able to simultaneously maintain leaf temperatures
below damaging extremes and avoid drought-induced embolism. Indeed,
previous studies have shown lower drought resistance in more exposed
trees
\citep{liu_effect_1993, suarez_factors_2004, scharnweber_confessions_2019}.
Unfortunately, collinearity between height and crown exposure in this
study (Fig. \textbf{2d}) makes it impossible to confidently partition
causality. Additional research comparing drought responses of early
successional and mature forest stands, along with short and tall
isolated trees, would be valuable for more clearly disentangling the
roles of tree height and crown exposure.

Belowground, taller trees would tend to have larger root systems
\citep{enquist_global_2002}, but this does not necessarily imply that
they have greater access to or reliance on deep soil-water resources
that may be critical during drought. While tree size can correlate with
the depth of water extraction \citep{brum_hydrological_2018}, the
linkage is not consistent. Shorter trees can vary broadly in the depth
of water uptake \citep{stahl_depth_2013}, and larger trees may allocate
more to abundant shallow roots that are beneficial for taking up water
from rainstorms \citep{meinzer_partitioning_1999}. Moreover, reliance on
deep soil-water resources can actually prove a liability during severe
and prolonged drought, as these can experience more intense water
scarcity relative to non-drought conditions
\citep{chitra-tarak_roots_2018}. In any case, the potentially greater
access to water did not override the disadvantage conferred by
height--and, in fact, greater moisture access in non-drought years
(here, higher TWI) appears to make trees more sensitive to drought
\citep{zuleta_drought-induced_2017, stovall_tree_2019}. This may be
because moister habitats would tend to support species and individuals
with more mesophytic traits
\citep{bartlett_drought_2016, mencuccini_ecological_2003, medeiros_extensive_2019},
potentially growing to greater heights (e.g.,
\citet{detto_hydrological_2013}), and these are then more vulnerable
when drought hits. The observed height-sensitivity of \(Rt\), together
with the lack of conferred advantage to large stature in drier
topographic positions, agrees with the concept that physiological
limitations to transpiration under drought shift from soil water
availability to the plant-atmosphere interface as forests age
\citep{bretfeld_plant_2018}, such that tall, dominant trees are the most
sensitive in mature forests. Again, additional research comparing
drought responses across forests with different tree heights and water
availability would be valuable for disentangling the relative importance
of above- and belowground mechanisms across trees of different size.

The development of tree-ring chronologies for the twelve most dominant
tree species at our site
\citep{helcoski_growing_2019, bourg_initial_2013} gave us the sample
size to compare historical drought responses across species (Fig.
\textbf{3}) and associated traits at a single site \citep[see
also][]{elliott_forest_2015}. Our study reinforced current understanding
(see Introduction) that wood density and \(LMA\) are not reliably linked
to drought resistance (Table 1). Contrary to previous studies in
temperate deciduous forests, we did not find an association between
xylem porosity and drought tolerance, as the two diffuse-porous species,
\emph{Liriodendron tulipifera} and \emph{Fagus grandifolia}, were at
opposite ends of the \(Rt\) spectrum (Fig. \textbf{3}). While the low
\(Rt\) of \emph{L. tulipifera} is consistent with other studies
\citep{elliott_forest_2015}, the high \(Rt\) of \emph{F. grandifolia}
contrasts with studies identifying diffuse porous species in general
\citep{elliott_forest_2015, kannenberg_linking_2019}, and the genus
\emph{Fagus} in particular \citep{friedrichs_species-specific_2009}, as
drought sensitive. There are two potential explanations for this
discrepancy. First, other traits can and do override the influence of
xylem porosity on drought resistance. Ring-porous species are restricted
mainly to temperate deciduous forests \citep{wheeler_variations_2007},
while highly drought-tolerant diffuse-porous species exist in other
biomes (\textbf{REFS}). \emph{Fagus grandifolia} had intermediate
\(\pi_{tlp}\) and low \(PLA_{dry}\) (Fig. \textbf{S4}), which would have
contributed to its drought resistance (Fig. \textbf{4}; see discussion
below). A second explanation of why \emph{F. grandifolia} trees at this
particular site had higher \(Rt\) is that the sampled individuals,
reflective of the population within the plot, are generally shorter and
in less-dominant canopy positions compared to most other species (Fig.
\textbf{S4}). The species, which is highly shade-tolerant, also has deep
crowns \citep{andersonteixeira_size-related_2015}, implying that a lower
proportion of leaves would be affected by harsher microclimatic
conditions at the top of the canopy under drought (Fig. \textbf{2}).
Thus, the high \(Rt\) of the sampled \emph{F. grandifolia} population
can be explained by a combination of fairly drought-resistant leaf
traits, shorter stature, and a buffered microenvironment.

Concerted measurement of tree-rings and leaf drought tolerance traits of
emerging importance
\citep{scoffoni_leaf_2014, bartlett_correlations_2016, medeiros_extensive_2019}
allowed novel insights into the role of drought tolerance traits in
shaping drought response. The finding that \(PLA_{dry}\) and
\(\pi_{tlp}\) can be useful for predicting drought responses of tree
growth (Fig. \textbf{4}; Table 1) is both novel and consistent with
previous studies linking these traits to habitat and drought tolerance.
Previous studies have demonstrated that \(\pi_{tlp}\) and \(PLA_{dry}\)
are physiologically meaningful traits linked to species distribution
along moisture gradients
\citep{marechaux_drought_2015, fletcher_evolution_2018, medeiros_extensive_2019, simeone_coupled_2019, rosas_adjustments_2019, zhu_leaf_2018},
and our findings indicate that these traits also influence drought
responses. Furthermore, the observed linkage of \(\pi_{tlp}\) to \(Rt\)
in this forest aligns with observations in the Amazon that \(\pi_{tlp}\)
is higher in drought-intolerant than drought-tolerant plant functional
type. Further, it adds support to the idea that this trait is useful for
categorizing and representing species' drought responses in models
\citep{powell_differences_2017}. Because both \(PLA_{dry}\) and
\(\pi_{tlp}\) can be measured relatively easily
\citep{bartlett_rapid_2012, scoffoni_leaf_2014}, they hold promise for
predicting drought growth responses across diverse forests. The
importance of predicting drought responses from species traits increases
with tree species diversity; whereas it is feasible to study drought
responses for all dominant species in most boreal and temperate forests
(e.g., this study), this becomes difficult to impossible for species
that do not form annual rings, and for diverse tropical forests.
Although progress is being made for the tropics
\citep{schongart_dendroecological_2017}, a full linkage of drought
tolerance traits to drought responses would be invaluable for
forecasting how little-known species and whole forests will respond to
future droughts
\citep{christoffersen_linking_2016, powell_differences_2017}.

As climate change drives increasing drought in many of the world's
forests
\citep{trenberth_global_2014, intergovernmental_panel_on_climate_change_climate_2015},
the fate of forests and their climate feedbacks will be shaped by the
biophysical and physiological drivers observed here. Our results,
consistent with other observations around the world, imply that the
tallest, most exposed trees will be most affected
\citep{bennett_larger_2015, stovall_tree_2019}. We show that, at least
within the mature forest studied here, the vulnerability conferred by
tall height and associated crown exposure outweigh any advantage of a
larger root system, even in drier microenvironments. This would suggest
that the drought responses of trees in mature forests are more strongly
differentiated along the size spectrum by their above- than below-ground
environment. The same may not be true of systems where short trees exist
outside of a buffered understory environment--\emph{i.e.}, open grown
trees or short-statured, early-successional forests. The latter appear
to be limited more strongly by root water access during drought
\citep{bretfeld_plant_2018}, and would also be dominated by species with
different traits. The earlier-successional species at our site
(\emph{Liriodendron tulipifera}, \emph{Quercus spp.}, \emph{Fraxinus
americana}) display a mix of traits conferring drought tolerance and
resistance (Table 2), while the late-successional \emph{Fagus
grandifolia} displayed high drought resistance, in part because it
exists primarily within a buffered microenvironment. Further research on
how leaf drought tolerance traits and drought vulnerability change over
the course of succession would be valuable for addressing how drought
tolerance changes as forests age
\citep[e.g.~][]{rodriguez-caton_long-term_2015}. In the meantime, the
results of this study advance our knowledge of the factors conferring
drought resistance in a mature forest, opening the door for more
accurate forecasting of forest responses to future drought.

\hypertarget{acknowledgements}{%
\subsubsection{Acknowledgements}\label{acknowledgements}}

We especially thank the numerous researchers who helped to collect the
data used here, in particular Jennifer C. McGarvey, Jonathan R.
Thompson, and Victoria Meakem for original collection and processing of
cores. Thanks also to Camila D. Medeiros for guidance on leaf drought
tolerance and functional trait measurements, Edward Brzostek's lab for
collaboration on leaf sampling, and Maya Prestipino for data collection.
This manuscript was improved based on helpful reviews by Mark Olson and
three anonymous reviewers. Funding for the establishment of the SCBI
ForestGEO Large Forest Dynamics Plot was provided by the Smithsonian-led
Forest Global Earth Observatory (ForestGEO), the Smithsonian
Institution, and the HSBC Climate Partnership. This study was funded by
ForestGEO, a Virginia Native Plant Society grant to KAT and AJT, and
support from the Harvard Forest and National Science Foundation which
supports the PalEON project (NSF EF-1241930) for NP.

\hypertarget{author-contribution}{%
\subsubsection{Author Contribution}\label{author-contribution}}

KAT, IM, and AJT designed the research. Tree-ring chronologies were
developed by RH under guidance of AJT and NP. Trait data were collected
by IM, JZ under guidance of NK and LS. Other plot data were collected by
IM, AS, EGA, and NB under guidance of EGA and WM. Data analyses were
performed by IM under guidance of KAT and VH. KAT and IM interpreted the
results. IM and KAT wrote the first draft of manuscript, and all authors
contributed to revisions.

\hypertarget{supplementary-information}{%
\subsubsection{Supplementary
Information}\label{supplementary-information}}

Table S1: Monthly Palmer Drought Severity Index (PDSI), and its rank
among all years between 1950 and 2009 (driest=1), for focal droughts.

Table S2: Species-specific bark thickness regression equations.

Table S3: Species-specific height regression equations.

Table S4. Individual tests of species traits as drivers of drought
resistance, where \(Rt\) is used as the response variable.

Table S5. Individual tests of species traits as drivers of drought
resistance, where \(Rt_{ARIMA}\) is used as the response variable.

Table S6. Summary of top full models for each drought instance, where
\(Rt\) is used as the response variable.

Table S7. Summary of top models for each drought instance, where
\(Rt_{ARIMA}\) is used as the response variable.

Figure S1. Time series of Palmer Drought Severity Index (PDSI) for the
2.5 years prior to each focal drought

Figure S2: Map of ForestGEO plot showing topographic wetness index and
location of cored trees

Figure S3: Distribution of reconstructed tree heights across drought
years.

Figure S4. Distribution of independent variables by species.

Figure S5. Comparison of \(Rt\) and \(Rt_{ARIMA}\) results, with
residuals, for each drought scenario

Figure S6. Visualization of best model, with data, for all droughts
combined.

  \bibliography{book.bib,packages.bib}

\end{document}
